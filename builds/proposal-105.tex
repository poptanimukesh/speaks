\documentclass[11pt]{article}
\usepackage[a4paper,top=1in,bottom=0.53in,left=0.53in,right=0.53in]{geometry}

\usepackage[T1]{fontenc}
\usepackage[utf8x]{inputenc}
\usepackage{authblk}
\usepackage{fancyhdr}
\usepackage{hhline}
%\renewcommand{\familydefault}{\sfdefault}

\usepackage{sansmath}
\usepackage{listingsutf8}
\lstset{inputencoding=utf8x,extendedchars=\true,basicstyle=\normalfont,columns=fullflexible,breaklines=true,breakindent=0em}

\setlength\parindent{0pt}

\fancypagestyle{fancyfirst}
{
	\renewcommand{\headrulewidth}{0pt}
	\fancyhf{}
	\chead{}
	\lhead{}
	\rhead{Project ID: 105
	}
	\lfoot{}
	\rfoot{}
}

\title{\vspace{3in}Double Shot Espresso
\\{\small Project Proposal}}
\author[]{Johnny Woo (johnnywoo@usc.edu)}\author[]{Phoebe Buffay (buffay@usc.edu)}\author[]{Frasier Crane (frasier@crane.com)}\author[]{Niles Crane (niles@crane.com)}
\affil[]{}
\date{}

\begin{document}
\maketitle
\thispagestyle{fancyfirst}
\pagebreak

{\bfseries What is your research question?}
\begin{lstlisting}[mathescape]
1
\end{lstlisting}
\hfill

{\bfseries Please list your specific research objectives.}
\begin{lstlisting}[mathescape]
1
\end{lstlisting}
\hfill

{\bfseries State the significance/background of the problem that your project addresses. Please cite a minimum of 3 references in AMA citation style used to support your background. (Approximately 300 words)}
\begin{lstlisting}[mathescape]
For many years, stigma has surrounded the use of marijuana in healthcare; however, recent studies have shown significant benefits of its use in the management of chronic pain. There is evidence that cannabinoid use as monotherapy and/or in conjunction with opiates lead to more significant pain relief, leading to less use of opioids and reduced side effects.[4]  This eventually leads to less patient hospitalizations/ length of stay and a reduction in healthcare costs. Additionally, medical marijuana use has shown great promise in the battle against the opioid crisis where many patients are struggling with addiction and abuse.[2]  The increased legalization of marijuana across the United States have led to community dispensaries that provide patients with an alternative approach to pharmacotherapy for chronic pain.[4] With an increase in cannabis use, there is little information about the economic impact that cannabis has on the health care system. Patients are now starting to use cannabis more frequently as opposed to their prescription pain medications, as over the years the perceived risk of marijuana has decreased.[1]  With the rise in cannabis legalization across the country, it is not impossible to see cannabis fall out of its current Schedule 1 drug category. This increase in cannabis affects the patients, providers, as well as payers in different ways. According to the World Health Organization, a patient requiring the marijuana equivalent of 1 to 2 marijuana cigarettes per day would need 0.5 to 1 oz of marijuana per month.[3]  Average marijuana market prices in Colorado and Washington are approximately \$10 per gram.[5]  By examining and comparing the costs associated with marijuana use versus prescription opioids for the treatment of chronic pain, patients with chronic pain may utilize this information to make the best treatment decisions.   Citations  Carliner, H., Brown, Q. L., Sarvet, A. L., \& Hasin, D. S. (2017). Cannabis use, attitudes, and legal status in the US: A review. Preventive medicine. https://www.sciencedirect.com/science/article/pii/S0091743517302554   Hill KP, Palastro MD, Johnson B, Ditre JW. Cannabis and Pain: A Clinical Review. Cannabis and Cannabinoid Res. 2017;2(1):96-104. doi:10.1089/can.2017.0017.  https://www.ncbi.nlm.nih.gov/pmc/articles/PMC5549367/  Hill KP. Medical Marijuana for Treatment of Chronic Pain and Other Medical and Psychiatric Problems A Clinical Review. JAMA J. 2015;313(24):2474-2483. https://jamanetwork.com/journals/jama/fullarticle/2338266  Lucas P. Cannabis as an Adjunct to or Substitute for Opiates in the Treatment of Chronic Pain. J Psychoactive Drugs. 2012;44(2):125-133. doi:10.1080/02791072.2012.684624  Hunt, P. \& Pacula, R.L. J Primary Prevent (2017) 38: 221. https://doi-org.libproxy2.usc.edu/10.1007/s10935-017-0471-x 
\end{lstlisting}
\hfill

{\bfseries Please indicate the study design that most closely describes the design of your study.}
\begin{lstlisting}[mathescape]
Randomized controlled trial
\end{lstlisting}
\hfill

{\bfseries Please indicate the area of practice to which your scholarly work most closely relates.}
\begin{lstlisting}[mathescape]

\end{lstlisting}
\hfill

{\bfseries Please describe your target population and sample size for your study. Consider and respond to the following: Who are your patients? What are the inclusion/exclusion criteria? How many patients do you aim to include?}
\begin{lstlisting}[mathescape]
1
\end{lstlisting}
\hfill

{\bfseries Please describe your data collection plan.}
\begin{lstlisting}[mathescape]
1
\end{lstlisting}
\hfill

{\bfseries Please describe your plan for statistical analysis of your data.}
\begin{lstlisting}[mathescape]
1
\end{lstlisting}
\hfill

{\bfseries Does your study require IRB approval? (You must discuss this with your mentor prior to submitting your proposal.)}
\begin{lstlisting}[mathescape]
Yes
\end{lstlisting}
\hfill

{\bfseries Please explain your rationale for why your study does or does not require IRB approval.}
\begin{lstlisting}[mathescape]
1
\end{lstlisting}
\hfill

{\bfseries If you believe your study does require IRB approval , what category of approval will you be seeking?}
\begin{lstlisting}[mathescape]
Expedited
\end{lstlisting}
\hfill

{\bfseries Please indicate your planned date for submission to IRB}
\begin{lstlisting}[mathescape]
2018-09-18
\end{lstlisting}
\hfill

{\bfseries Please describe the specific roles and responsibilities of each student group member for this scholarly project. (It is not reasonable or acceptable to indicate that every member of your group will do every part of the project equally.)}
\begin{lstlisting}[mathescape]
1
\end{lstlisting}
\hfill

{\bfseries Briefly describe the general timeline for when you will complete the following items by entering the Month and Year.}
\begin{lstlisting}[mathescape]
Prepare data collection table
2

Submit to IRB (if applicable)
1

Start data collection
1

Finish data collection
1

Start data analysis
1

Finish data analysis
1

Start poster draft
1

Finish poster draft
1
\end{lstlisting}
\hfill

{\bfseries Why did you select the Project Advisor(s) for this project?}
\begin{lstlisting}[mathescape]
1
\end{lstlisting}
\hfill

{\bfseries Which of the following best describes how you came to work with your Project Advisor(s)?}
\begin{lstlisting}[mathescape]
IPPE
\end{lstlisting}
\hfill

\pagebreak
\begin{center}
	{\bfseries\large Project Advisor Section}
\end{center}

{\bfseries Thank you for agreeing to mentor one or more of our USC students on their PharmD Scholarly Project.}\\

{\bfseries Which of the following best describes your highest academic degree?}
\begin{lstlisting}[mathescape]
PharmD
\end{lstlisting}
\hfill

{\bfseries Which of the following describes your area of practice/work? Please check ALL that apply.}
\begin{lstlisting}[mathescape]

\end{lstlisting}
\hfill

{\bfseries Have you ever submitted research to the institutional review board (IRB)?}
\begin{lstlisting}[mathescape]

\end{lstlisting}
\hfill

{\bfseries What is your affiliation with the institution where the research is being conducted?}
\begin{lstlisting}[mathescape]

\end{lstlisting}
\hfill

{\bfseries Please briefly describe your previous scholarly work including research, quality improvement initiatives, publications, etc.}
\begin{lstlisting}[mathescape]

\end{lstlisting}
\hfill

\end{document}